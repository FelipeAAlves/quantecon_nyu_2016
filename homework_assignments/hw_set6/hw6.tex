\documentclass[11pt,twoside,fleqn,reqno,a4paper]{amsart}

%---------------------------------------------------------------------------------------------------------------------
%...................................................... PREAMBLE .....................................................
%---------------------------------------------------------------------------------------------------------------------
\usepackage{natbib,amsmath,amsfonts,fullpage,hyphenat,booktabs,enumerate,graphicx,float,subfig} 
\usepackage[colorlinks,linkcolor=black,citecolor=blue,urlcolor=black]{hyperref} 

\usepackage{float,mathpazo,enumerate,tikz,pdfpages}
\usepackage[latin1]{inputenc}
\makeatletter
\def\subsection{\@startsection{subsection}{3}%
  \z@{.5\linespacing\@plus.7\linespacing}{.1\linespacing}%
  {\normalfont\itshape}}
%Set author not to be UPPERCASE  
\def\@setauthors{%
  \begingroup
  \def\thanks{\protect\thanks@warning}%
  \trivlist
  \centering\normalsize \@topsep30\p@\relax
  \advance\@topsep by -\baselineskip
  \item\relax
  \author@andify\authors
  \def\\{\protect\linebreak}%
%  \MakeUppercase{\authors}%
  \authors%
  \ifx\@empty\contribs
  \else
    ,\penalty-3 \space \@setcontribs
    \@closetoccontribs
  \fi
  \endtrivlist
  \endgroup
}
\makeatother

%- - - -- - - -- - - -- - - -- - - -- - - -- - - -- - - -- 
%--------------- Line Spec -------------
%- - - -- - - -- - - -- - - -- - - -- - - -- - - -- - - -- 
\usepackage[textwidth = 160mm, textheight = 220mm]{geometry}
% \usepackage{geometry}
\usepackage{setspace}
\setstretch{1.25}
\setlength{\parskip}{0.5\baselineskip}
\setlength\parindent{0pt}

\setlength{\voffset}{-0.50in}
\setlength{\headsep}{25pt}
\linespread{1.5}

\usepackage{titling}
\setlength{\droptitle}{-5em}     % Eliminate the default vertical space
\addtolength{\droptitle}{-20pt} 

\numberwithin{equation}{section}
%- - - -- - - -- - - -- - - -- - - -- - - -- - - -- - - -- - - -- - - -- - - -- - - -- - - -- - - -- - - -
%---------------------------------- Cabeçalho ----------------------------------------------
%- - - -- - - -- - - -- - - -- - - -- - - -- - - -- - - -- - - -- - - -- - - -- - - -- - - -- - - -- - - -
\usepackage{fancyhdr}
\pagestyle{fancy}
\lhead[\sc Class]{}
\chead[]{\sc NYU Stern}
\rhead[\emph{Felipe Alves}]{}


%- - - -- - - -- - - -- - - -- - - -- - - -- - - -- - - -- - - -- - - -- - - -- - - -- - - -- - - -- - - -
%---------------------------------- MaTH ----------------------------------------------
%- - - -- - - -- - - -- - - -- - - -- - - -- - - -- - - -- - - -- - - -- - - -- - - -- - - -- - - -- - - -

\DeclareMathOperator*{\Max}{Max}
\DeclareMathOperator*{\Min}{Min}
\DeclareMathOperator*{\argmax}{argmax}

\newcommand{\matr}[1]{\mathbf{#1}} % undergraduate algebra version
%\newcommand{\matr}[1]{#1}          % pure math version
%\newcommand{\matr}[1]{\bm{#1}}     % ISO complying version
\newcommand*{\defeq}{\mathrel{\vcenter{\baselineskip0.5ex \lineskiplimit0pt
                     \hbox{\scriptsize.}\hbox{\scriptsize.}}}%
                     =}

%- - - -- - - -- - - -- - - -- - - -- - - -- - - -- - - -- - - -- - - -- - - -- - - -- - - -- - - -- - - -
%---------------------------------- Figure ----------------------------------------------
%- - - -- - - -- - - -- - - -- - - -- - - -- - - -- - - -- - - -- - - -- - - -- - - -- - - -- - - -- - - -
\usepackage{graphicx}%,showframe}
\usepackage{float}
\usepackage[]{subfig}
\usepackage[font=small,labelfont=sc,position=top]{caption}

\author{Felipe Alves \ (NYU Stern)}
\title{\normalfont\bf Stachuski \\ Problem Set 6 }
\date{}

%\begin{document} 
%\maketitle{} \begin{abstract} \end{abstract} 
\begin{document}
\maketitle
%========================================================================
%------------------------------------ Setup -------------------------------------------
%========================================================================

%- - - -- - - -- - - -- - - -- - - -- - - -- - - -- - - -- - - -- - - -- - - -- - - -- - - -- - - -- - - -- - 
%------------------------------------------------ Question 1 ------------------------------------------------
%- - - -- - - -- - - -- - - -- - - -- - - -- - - -- - - -- - - -- - - -- - - -- - - -- - - -- - - -- - - -- - 
\section*{Exercise 1}
\setcounter{section}{3}
\begin{em}
Show that $\matr{X}'\matr{X}$ is invertible.
\end{em}

%========================================================================
\textbf{Answer}:
%========================================================================

Let $0 \ne \alpha \in \mathbb{R}^k$. Then
\begin{align*}
\alpha' ( X'X ) \alpha & = ( X\alpha )' ( X \alpha ) \\
                       & = \left( \sum_{j=1}^k \alpha_j x_j \right)'   \underbrace{\left( \sum_{j=1}^k \alpha_j x_j \right)}_{z} \\
                       & = \sum_{i=1}^n z_i^2 \ge 0                 
\end{align*}
where $x_j$ is the $j$-th column of matrix $X$. 
But since $\{x_1,\ldots, x_k\}$ is LI, $ z $ as a linear combination of columns of $X$ is nonzero  
and therefore we have inequality holding strictly. 
This implies that $ X'X $ is \emph{positive definite}. 

Hence, all eigenvalues of $ ( X'X ) $ are positive. 
To see this, let $ \lambda_j,z_j $ be an eigenvalue, eigenvector pair of $ ( X'X ) $. Then
\[ 
    \lambda_j z_j = ( X'X )z_j \Rightarrow \lambda_j z_j'z_j = z_j' ( X'X ) z_j >0
\]
but $\lambda_j \lVert z \rVert^2>0 \Leftrightarrow \lambda_j >0$. 
Taking as given the result regarding the relaton of determinant of a matrix and its eigenvalues - $ \text{det} A = \Pi_{j=1}^k \lambda_j $ - 
we get that determinant of $ X'X $ is positive, therefore invertible.

\vfill \
%- - - -- - - -- - - -- - - -- - - -- - - -- - - -- - - -- - - -- - - -- - - -- - - -- - - -- - - -- - - -- - 
%------------------------------------------------ Question 2 ------------------------------------------------
%- - - -- - - -- - - -- - - -- - - -- - - -- - - -- - - -- - - -- - - -- - - -- - - -- - - -- - - -- - - -- - 
\section*{Exercise 2}
\setcounter{section}{3}
\begin{em}
Let $ \matr{P} = \matr{X} ( \matr{X}' \matr{X})^{-1} \matr{X}' $. Show that if $k=n$, then $ \matr{P} $
is the identity.
\end{em}

%========================================================================
\textbf{Answer}:
%========================================================================

If $ k=n $, then $X$ is square and, because we assumed that it has LI columns, is invertible. 
Since $X$ invertible implies $ X'$ invertible%
\footnote{The fact that $X$ is investible mean that exists $X^{-1}$ such that $XX^{-1} = X^{-1} X = I $
but in that case $( X^{-1} )' X' = X' ( X^{-1} )' =I \Rightarrow ( X^{-1} )' = ( X' )^{-1}$}
%
we have
\[ 
    P = X( X'X )^{-1} X' = XX^{-1} ( X' )^{-1}X' = I
\]
The result is also intuitive. From the orthogonal projection theorem we know that $ Py $ is the vector in span$X$ which best
approximates $y$. Here, span$X=\mathbb{R}^n$, so $Py = y \ \forall y\in\mathbb{R}^n $, which implies $P=I$
%- - - -- - - -- - - -- - - -- - - -- - - -- - - -- - - -- - - -- - - -- - - -- - - -- - - -- - - -- - - -- - 
%------------------------------------------------ Question 1 ------------------------------------------------
%- - - -- - - -- - - -- - - -- - - -- - - -- - - -- - - -- - - -- - - -- - - -- - - -- - - -- - - -- - - -- - 
\section*{Exercise 3}
\setcounter{section}{3}
\begin{em}
Show that the projection of $ \matr{y} \in \mathbb{R}^n $ onto span$\{1\}$ is the mean of elements of $y$.
\end{em}

%========================================================================
\textbf{Answer}:
%========================================================================

Let $S= \text{span } \{ \mathbf{1} \}$. From the lecture notes, we have that $ \text{proj } S$ can be represented
in a matrix form as $\matr{P} = \mathbf{1} (\mathbf{1}'\mathbf{1})^{-1} \mathbf{1}' $. Therefore the projection 
of $y \in \mathbb{R}^n$ onto $S$ is given by
\begin{align*}
\matr{P} y  & = \mathbf{1} (\mathbf{1}'\mathbf{1})^{-1} \mathbf{1}'y \\
            & = \mathbf{1} (n)^{-1} \sum_{i=1}^n y_i \\
            & = \begin{bmatrix}
                n^{1} \sum_{i=1}^n y_i \\ \vdots \\ n^{1} \sum_{i=1}^n y_i
                \end{bmatrix}  
\end{align*}

%- - - -- - - -- - - -- - - -- - - -- - - -- - - -- - - -- - - -- - - -- - - -- - - -- - - -- - - -- - - -- - 
%------------------------------------------------ Question 1 ------------------------------------------------
%- - - -- - - -- - - -- - - -- - - -- - - -- - - -- - - -- - - -- - - -- - - -- - - -- - - -- - - -- - - -- - 
\section*{Exercise 4}
\setcounter{section}{3}
\begin{em}
Let $ y\in \mathbb{R}^n $ and $ \matr{X} \in \mathcal{M}(n\times k) $ have linearly independent columns. Let
$S = \text{span } \{x_1, \ldots, x_k\}$, $ \matr{P} = \text{proj } S$ and $ \matr{M} = \matr{I} - \matr{P} $. 
Show that if $ x_1 = \mathbf{1} $, then elements of $\hat{u} = \matr{M}y$ sum to zero.
\end{em} 

%========================================================================
\textbf{Answer}: \\
%========================================================================
By definition of $ \matr{M} $, we can rewrite $ \hat{u} = y - \matr{P}y$. From the \textbf{Orthogonal Projection
Theorem} we have $ y - \matr{P}y \bot S$. In particular, $\hat{u}$ must be orthogonal to $ \mathbf{1}$, since $\mathbf{1}\in S$. 
Therefore, the result follows from the realization of $\sum_{i=1}^n \hat{u}_i = \langle \mathbf{1}, \hat{u} \rangle = 0$.


%- - - -- - - -- - - -- - - -- - - -- - - -- - - -- - - -- - - -- - - -- - - -- - - -- - - -- - - -- - - -- - 
%------------------------------------------------ Question 1 ------------------------------------------------
%- - - -- - - -- - - -- - - -- - - -- - - -- - - -- - - -- - - -- - - -- - - -- - - -- - - -- - - -- - - -- - 
\section*{Exercise 5}
\setcounter{section}{3}
\begin{em}
Show that if $S$ is a nonempty subset of $\mathbb{R}^n$, then $ S\cap S^{\bot} = \{0\}$.
\end{em}

%========================================================================
\textbf{Answer}:
%========================================================================

The following is true
\[  
    x\in S\cap S^{\bot} \Rightarrow \langle x,x \rangle = \lVert x\rVert^2 = 0 \Leftrightarrow x = 0  
\] 
where the last inequality follows from the properties of the norm induced by $ \langle \cdot,\cdot \rangle $.

However, from the asusmption on the statement only, we cannot be sure that $S\cap S^{\bot} \ne \emptyset$ - we would if $ S$
was assumed to be a linear subspace. So, we either have $ S\cap S^{\bot} = \emptyset $ if $ 0 \not\in S $
or the derised result if we add the assumption $0\in S$.

\bibliographystyle{apalike} \bibliography{all}{} \end{document}